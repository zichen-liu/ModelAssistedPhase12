\documentclass[notitlepage]{article}

\usepackage{amsmath}
\usepackage{listings}
\usepackage[svgnames]{xcolor}
\lstset{language=R,
    basicstyle=\small\ttfamily,
    stringstyle=\color{DarkGreen},
    otherkeywords={0,1,2,3,4,5,6,7,8,9},
    morekeywords={TRUE,FALSE},
    deletekeywords={data,frame,is,length,as,character},
    keywordstyle=\color{black},
    commentstyle=\color{DarkGreen},
}

% Title Page
\begin{document}
%\maketitle
\begin{center}
{\bf \large
 Source code for manuscript ``Comparative Review of Novel Model-Assisted Designs for
Phase I/II Clinical Trials''
}\end{center}

\begin{center}
{
 Haolun Shi, Ruitao Lin, Xiaolei Lin
}\end{center}


\section{Introduction}
The code has been written using R-3.6.3 (platform:  x86\_64-w64-mingw32 x64 (64-bit) with
package versions data.table\_1.14.2 and Iso\_0.0-18.1.

To reproduce the tables in the manuscript, run the main.r.
The script will produce all the intermediate results (in csv format) and Figures.
A detailed description of each R code are given as follows.
\begin{itemize}
\item[$\diamond$] main.r: The script will produce all the intermediate results (in csv format) and Figures.
\item[$\diamond$] Figure1.r: The script will reproduce Figure 1 in the manuscript.
\item[$\diamond$] Figure2.r: The script will reproduce Figure 2 in the manuscript.
\item[$\diamond$] Figure3-8.r: The script will reproduce Figures 3-8 in the manuscript.
\item[$\diamond$] boin12.R: The script for generating operating characteristics of the BOIN12 design.
\item[$\diamond$] BOINET.R: The script for generating operating characteristics of the BOIN-ET design.
\item[$\diamond$] efftox.r: The script for generating operating characteristics of the EffTox design.
\item[$\diamond$] Ji3.R: The script for generating operating characteristics of the Joint i3+3 design.
\item[$\diamond$] PITE.R: The script for generating operating characteristics of the PRINTE design.
\item[$\diamond$] STEIN.R: The script for generating operating characteristics of the STEIN design.
\item[$\diamond$] TEPI.R: The script for generating operating characteristics of the TEPI design.
\item[$\diamond$] uTPI.R: The script for generating operating characteristics of the uTPI design.
\item[$\diamond$] extractavg.r: The script for extracting and averaging the operating characteristics of the model-assisted design into a single csv files named ``simres.csv''.
\item[$\diamond$] extractavg2.r: The script for extracting and averaging the operating characteristics of the model-assisted design  and EffTox design into a single csv files named ``simres.csv''.
\item[$\diamond$] extractavg\_sens.r: The script for extracting and averaging the operating characteristics of the model-assisted design into a single csv files named ``simres\_sensitivity.csv'' for sensitivity analyses.
\end{itemize}
Note that the running time is long and we have stored all the intermediate results under the folders.


Description of the folder structures are as follows.
de are given as follows.
\begin{itemize}
\item[$\diamond$] NDOSE3/: The folder contains the intermediate results for simulation with the number of dose level $J=3$.
\item[$\diamond$] NDOSE5/: The folder contains the intermediate results for simulation with the number of dose level $J=5$.
\item[$\diamond$] NDOSE5\_START2/: The folder contains the intermediate results for simulation with the number of dose level $J=5$ and starting dose level 2.
\item[$\diamond$] NDOSE10/: The folder contains the intermediate results for simulation with the number of dose level $J=10$.
\item[$\diamond$] sensitivity/: The folder contains the intermediate results for simulation on sensitivity analyses. The subfolder within corresponds to the simulation with a certain target efficacy rate $\zeta_E$ and target toxicity rate $\phi_T$ (subfolder name is simply the $\zeta_E$ and $\phi_T$ connected by a ``\_''.
\end{itemize}
Each folder contains a number of subfolders indexed from 1 to $J$, which correspond to the scenario where the OBD is fixed at the index. For example, the subfolder NDOSE3/2 corresponds to the simulation scenario where the OBD is fixed at the second dose level in the simulation where there are $J=3$ dose levels.


To reproduce the tables in the manuscript, we use Excel to create the formatted tables and use Excel2Latex for converting the tables into latex. We list them as follows.
\begin{itemize}
\item[$\diamond$] NDOSE5/Table3.xlsm
\item[$\diamond$] NDOSE5\_START2/Table4.xlsm
\item[$\diamond$] NDOSE3/Table5.xlsm
\item[$\diamond$] NDOSE10/Table6.xlsm
\item[$\diamond$] sensitivity/Table7.xlsm
\end{itemize}
Tables 1 and 2 are text-based tables and contain no numbers, thus are omitted.

For questions, comments or remarks about the code, please contact the first author of the paper.

\section{Additional Guidance for User-Specified Scenarios}
 In practice, a user may prefer to specify parameters that are different from the ones in the manuscript. Moreover, the user may prefer to set up fixed scenarios for comparison.
 In this section, we provide additional guidance on how to compare the user-specified designs and setting up user-specified ground truth.

 Note that this section only serve as an extra guidance note for readers that are interested to use the code to carry on their own simulation studies. This section does not correspond to or reproduce any of the results in the manuscript. The main.r is already suffices to reproduce all tables and figures.

 \subsection{Setting Up Ground Truth: Random Scenarios}
 In the case of user-specified random scenarios, the user may simply set the global variables \texttt{TARGET.E} and  \texttt{TARGET.T} before running the simulations.
 For example, if a user prefer to generate random scenarios with $\zeta_T = 0.25$ and $\phi_T = 0.35$ and evaluate the operating characteristics of BOIN12 design with 5 dose levels and number of cohorts being 30, the following code can be used.
 \begin{lstlisting}[language=R]
PATH = getwd()

NDOSE <<- 5
dir.create(paste0(PATH,'/NDOSE5/'), showWarnings = FALSE)
setwd(paste0(PATH,'/NDOSE5/'))

for(i in 1:NDOSE){
dir.create(paste0(i,'/'), showWarnings = FALSE)
}

STARTD <<- 1
TARGET.E <<- 0.25
TARGET.T <<- 0.35
SSIZERANGE <<- c(30)
source(paste0(PATH,'/boin12.R'))
\end{lstlisting}

 \subsection{Setting Up Ground Truth: Fixed Scenarios}
 In the case of user-specified fixed scenarios, a fixed probability vectors needs to be supplied for the toxicity probability and efficacy probability. To do this, the user can simply replace the \texttt{simprob()} function within the source codes of the designs.

 For example, if a user prefer to evaluate operating characteristics  where the toxicity and efficacy probabilities for the five doses are $(0.1,0.2,0.3,0.4,0.4)$ and $(0.4,0.5,0.6,0.6,0.6)$, the following code can be used to replace the \texttt{simprob()} function.
 of cohorts being 30, the following code can be used to  replace the \texttt{simprob()} function within the source codes of the designs.
 \begin{lstlisting}[language=R]
simprob<-function(ndose,targetE,targetT,u1,u2 ,randomtype){
  pT = c(0.1,0.2,0.3,0.4,0.4)
  pE = c(0.4,0.5,0.6,0.6,0.6)
  obd = 3
  mtd = 2
  return(list(pE = pE,pT=pT,obd=obd,mtd=mtd))
}
\end{lstlisting}

  \subsection{Setting Design Parameters}
  We describe how to set up the design parameters in the following subsections.
  For BOIN-ET and EffTox designs, many of the key parameters are calculated from external software/codes. The code for BOIN-ET can be obtained upon request from Takeda, K., Taguri, M., and Morita, S, who are the authors of the paper. As for the software for implementing the EffTox design, we refer the readers to the website \texttt{https://biostatistics.mdanderson.org/SoftwareDownload/}.

 \subsubsection{BOIN12 Design Parameters}
 We describe the design parameters that underlies the arguments for the function \texttt{get.oc.obd()} in the BOIN12.R as follows.
 \begin{itemize}
\item[$\diamond$] \texttt{targetT}: target toxicity probability, i.e., the $\phi_T$ in the manuscript.
\item[$\diamond$] \texttt{targetE}: lower limit for the efficacy rate, i.e., the $\zeta_E$ in the manuscript.
\item[$\diamond$] \texttt{ncohort}: the number of cohorts.
\item[$\diamond$] \texttt{cohortsize}: the size of a cohort.
\item[$\diamond$] \texttt{startdose}: the starting dose level.
\item[$\diamond$] \texttt{cutoff.eli.T}: the posterior probability cutoff for toxicity dose elimination rule.
\item[$\diamond$] \texttt{cutoff.eli.E}: the posterior probability cutoff for futility dose elimination rule.
\item[$\diamond$] \texttt{u1}: the utility parameter $w_{11}$, in the scale of 0 to 100.
\item[$\diamond$] \texttt{u2}: the utility parameter $w_{00}$, in the scale of 0 to 100.
\item[$\diamond$] \texttt{ntrial}: the number of random trial replications.
\item[$\diamond$] \texttt{utilitytype}: a overriding argument for controlling the type of utility parameters. If set as 1, then $(w_{11},w_{00}) = (0.6,0.4)$. If set as 2, $(w_{11},w_{00}) = (1,0)$. Otherwise, the user-specified values for \texttt{u1} and \texttt{u2} are used.
\end{itemize}
 The early stage sample size cutoff is set as the recommended default values, and rarely needs to be modified. The rest of the design parameters are intermediate and depend upon the user-specified parameters listed above.

 If the user wish to specify utility structure, he may simply modify the values in the following section within the source code and set \texttt{utilitytype = 1}.
\begin{lstlisting}[language=R]
if (utilitytype==1){
u1 = 60
u2 = 40
}
\end{lstlisting}

 \subsubsection{uTPI Design Parameters}
 We describe the design parameters that underlies the arguments for the function \texttt{get.oc()} in the uTPI.R as follows.
 \begin{itemize}
\item[$\diamond$] \texttt{targetT}: target toxicity probability, i.e., the $\phi_T$ in the manuscript.
\item[$\diamond$] \texttt{targetE}: lower limit for the efficacy rate, i.e., the $\zeta_E$ in the manuscript.
\item[$\diamond$] \texttt{ncohort}: the number of cohorts.
\item[$\diamond$] \texttt{cohortsize}: the size of a cohort.
\item[$\diamond$] \texttt{startdose}: the starting dose level.
\item[$\diamond$] \texttt{cutoff.eli.T}: the posterior probability cutoff for toxicity dose elimination rule.
\item[$\diamond$] \texttt{cutoff.eli.E}: the posterior probability cutoff for futility dose elimination rule.
\item[$\diamond$] \texttt{ntrial}: the number of random trial replications.
\item[$\diamond$] \texttt{utilitytype}: a overriding argument for controlling the type of utility parameters. If set as 1, then $(w_{11},w_{00}) = (0.6,0.4)$. If set as 2, $(w_{11},w_{00}) = (1,0)$.
\end{itemize}
 The early stage sample size cutoff is set as the recommended default values, and rarely needs to be modified. The rest of the design parameters are intermediate and depend upon the user-specified parameters listed above.

If the user wish to specify utility structure, he may simply modify the values in the following section within the source code and set \texttt{utilitytype = 1}.
\begin{lstlisting}[language=R]
if (utilitytype==1){
u1 = 60
u2 = 40
}
\end{lstlisting}
 \subsubsection{TEPI Design Parameters}
 We describe the design parameters that underlies the arguments for the function \texttt{get.oc()} in the TEPI.R as follows.
 \begin{itemize}
\item[$\diamond$] \texttt{targetT}: target toxicity probability, i.e., the $\phi_T$ in the manuscript.
\item[$\diamond$] \texttt{targetE}: lower limit for the efficacy rate, i.e., the $\zeta_E$ in the manuscript.
\item[$\diamond$] \texttt{ncohort}: the number of cohorts.
\item[$\diamond$] \texttt{cohortsize}: the size of a cohort.
\item[$\diamond$] \texttt{startdose}: the starting dose level.
\item[$\diamond$] \texttt{cutoff.eli.T}: the posterior probability cutoff for toxicity dose elimination rule.
\item[$\diamond$] \texttt{cutoff.eli.E}: the posterior probability cutoff for futility dose elimination rule.
\item[$\diamond$] \texttt{ntrial}: the number of random trial replications.
\item[$\diamond$] \texttt{utilitytype}: a overriding argument for controlling the type of utility parameters. If set as 1, then $(w_{11},w_{00}) = (0.6,0.4)$. If set as 2, $(w_{11},w_{00}) = (1,0)$.
\end{itemize}
Note that the TEPI design requires a mapping table for dose assignment decision. These can be modified and specified in line 220 - 223 as follows (here, we set the table to be hinge upon the target efficacy and toxicity probability).
\begin{lstlisting}[language=R]
effint_l<-c(0,TARGET.E,TARGET.E+0.2,TARGET.E+0.4)
effint_u<-c(TARGET.E,TARGET.E+0.2,TARGET.E+0.4,1)
toxint_l<-c(0,0.15,TARGET.T,TARGET.T+0.05)
toxint_u<-c(0.15,TARGET.T,TARGET.T+0.05,1)
\end{lstlisting}
The 4 variables in the code above set up the boundaries within the mapping tables. Setting \texttt{TARGET.E} to be 0.2 and \texttt{TARGET.T} to be 0.3 reproduce Table 1 in the manuscript.

If the user wish to specify utility structure, he may simply modify the values in the following section within the source code and set \texttt{utilitytype = 1}.
\begin{lstlisting}[language=R]
if (utilitytype==1){
u1 = 60
u2 = 40
}
\end{lstlisting}


 \subsubsection{PRINTE Design Parameters}
 We describe the design parameters that underlies the arguments for the function \texttt{PITE\_sim()} in the  PITE.R as follows.
 \begin{itemize}
\item[$\diamond$] \texttt{pT}: target toxicity probability, i.e., the $\phi_T$ in the manuscript.
\item[$\diamond$] \texttt{pE}: the target  efficacy probability, i.e., the $\phi_E$ in the manuscript.
\item[$\diamond$] \texttt{zetaE}: lower limit for the efficacy rate, i.e., the $\zeta_E$ in the manuscript.
\item[$\diamond$] \texttt{ncohort}: the number of cohorts.
\item[$\diamond$] \texttt{cohortsize}: the size of a cohort.
\item[$\diamond$] \texttt{startdose}: the starting dose level.
\item[$\diamond$] \texttt{eps1}: Width of the subrectangle $\epsilon$.
\item[$\diamond$] \texttt{eps2}: Width of the subrectangle $\epsilon$.
\item[$\diamond$] \texttt{psafe}: the posterior probability cutoff for toxicity dose elimination rule.
\item[$\diamond$] \texttt{pfutility}: the posterior probability cutoff for futility dose elimination rule.
\item[$\diamond$] \texttt{nsimul}: the number of random trial replications.
\item[$\diamond$] \texttt{utilitytype}: a overriding argument for controlling the type of utility parameters. If set as 1, then $(w_{11},w_{00}) = (0.6,0.4)$. If set as 2, $(w_{11},w_{00}) = (1,0)$.
\end{itemize}
If the user wish to specify utility structure, he may simply modify the values in the following section within the source code and set \texttt{utilitytype = 1}.
\begin{lstlisting}[language=R]
if (utilitytype==1){
u1 = 60
u2 = 40
}
\end{lstlisting}



 \subsubsection{Joint i3+3 Design Parameters}
 We describe the design parameters that underlies the arguments for the function \texttt{PITE\_sim()} in the  PITE.R as follows.
 \begin{itemize}
\item[$\diamond$] \texttt{pT}: target toxicity probability, i.e., the $\phi_T$ in the manuscript.
\item[$\diamond$] \texttt{pE}: the target  efficacy probability, i.e., the $\phi_E$ in the manuscript.
\item[$\diamond$] \texttt{zetaE}: lower limit for the efficacy rate, i.e., the $\zeta_E$ in the manuscript.
\item[$\diamond$] \texttt{ncohort}: the number of cohorts.
\item[$\diamond$] \texttt{cohortsize}: the size of a cohort.
\item[$\diamond$] \texttt{startdose}: the starting dose level.
\item[$\diamond$] \texttt{eps1}: Width of the subrectangle $\epsilon$.
\item[$\diamond$] \texttt{eps2}: Width of the subrectangle $\epsilon$.
\item[$\diamond$] \texttt{psafe}: the posterior probability cutoff for toxicity dose elimination rule.
\item[$\diamond$] \texttt{pfutility}: the posterior probability cutoff for futility dose elimination rule.
\item[$\diamond$] \texttt{nsimul}: the number of random trial replications.
\item[$\diamond$] \texttt{utilitytype}: a overriding argument for controlling the type of utility parameters. If set as 1, then $(w_{11},w_{00}) = (0.6,0.4)$. If set as 2, $(w_{11},w_{00}) = (1,0)$.
\end{itemize}
If the user wish to specify utility structure, he may simply modify the values in the following section within the source code and set \texttt{utilitytype = 1}.
\begin{lstlisting}[language=R]
if (utilitytype==1){
u1 = 60
u2 = 40
}
\end{lstlisting}




 \subsubsection{STEIN Design Parameters}
 We describe the design parameters that underlies the arguments for the function \texttt{get.oc()} in the  STEIN.R as follows.
 \begin{itemize}
\item[$\diamond$] \texttt{targetT}: target toxicity probability, i.e., the $\phi_T$ in the manuscript.
\item[$\diamond$] \texttt{targetE}: lower limit for the efficacy rate, i.e., the $\zeta_E$ in the manuscript.
\item[$\diamond$] \texttt{psi1}: the highest efficacy probability deemed as inefficacious, i.e., the $\psi_{1,E}$ in the manuscript.
\item[$\diamond$] \texttt{psi2}: the lowest efficacy probability deemed as highly promising, i.e., the $\psi_{2,E}$ in the manuscript.
\item[$\diamond$] \texttt{ncohort}: the number of cohorts.
\item[$\diamond$] \texttt{cohortsize}: the size of a cohort.
\item[$\diamond$] \texttt{startdose}: the starting dose level.
\item[$\diamond$] \texttt{cutoff.eli.T}: the posterior probability cutoff for toxicity dose elimination rule.
\item[$\diamond$] \texttt{cutoff.eli.E}: the posterior probability cutoff for futility dose elimination rule.
\item[$\diamond$] \texttt{ntrial}: the number of random trial replications.
\item[$\diamond$] \texttt{utilitytype}: a overriding argument for controlling the type of utility parameters. If set as 1, then $(w_{11},w_{00}) = (0.6,0.4)$. If set as 2, $(w_{11},w_{00}) = (1,0)$.
\end{itemize}


If the user wish to specify utility structure, he may simply modify the values in the following section within the source code and set \texttt{utilitytype = 1}.
\begin{lstlisting}[language=R]
if (utilitytype==1){
u1 = 60
u2 = 40
}
\end{lstlisting}

The value of the design parameter $\eta_1$ is calculated from \texttt{psi1} and \texttt{psi2} at line 214 in the source code as follows. Note that in a comparison studies, this value needs to be aligned to be close to the $\phi_E$ in the joint i3+3 and PRINTE design (or \texttt{pE} as the functional argument).
\begin{lstlisting}[language=R]
psi<-log((1-psi1)/(1-psi2))/log(psi2*(1-psi1)/psi1/(1-psi2))
\end{lstlisting}








\end{document}
